\documentclass[12pt, a4paper, twoside]{article} %twoside is for fancyhdr





%--------------------------------------------------------------------------%
%--------------------------------------------------------------------------%
% ------------------------------  PACKAGES --------------------------------%
%--------------------------------------------------------------------------%
%--------------------------------------------------------------------------%






% ----------------------  GENERIC DOCUMENT SETTINGS -----------------------%
%--------------------------------------------------------------------------%
\usepackage{geometry}  
	\geometry{top = 3.5cm, bottom = 3cm, left = 2cm, right = 2cm, headsep = 1.4cm, footskip = 1.5cm}
\usepackage[skip = 7pt, indent = 0pt]{parskip} % paragraph settings
\usepackage{titlesec}%first parameter to set specific indentation to each section
					 %second and third paarmeters to set before and after spacing
					 % in section, subsection y subsubsection
    \titlespacing\section{0pt}{0.6cm}{0.6cm}
    \titlespacing\subsection{0pt}{0.5cm}{0.5cm}
    \titlespacing\subsubsection{0pt}{12pt plus 4pt minus 2pt}{5pt plus 2pt minus 2pt}
\usepackage{microtype} %improves the justification, and therefore the visualization of the text
\usepackage{times} %font family to times new roman
%%\usepackage{fontspec}%font-family
	%%\setmainfont{Verdana} %FALSE POSITIVE ERROR
	
\usepackage{hyperref} %links: internal & external
\hypersetup{ %for all kinds of links. It has more parameters but
	% we are just going to use the most basic ones
    colorlinks = true,
    linkcolor = blue, % internal links
    urlcolor = blue, % external links
    filecolor = green % link to your files
} % ------------- USE \hyperlink for internal link and \href for external ------------------%

\usepackage{fancyhdr} % To create custom headers and footnotes
%\fancypagestyle{plain}{ % este es un tipo de header y footer. Lo he llamado plain. 
%	\fancyhead{} % clear header
%	
%	\fancyhead[L]{\nouppercase\leftmark}
%	\fancyhead[RE]{\text{Trabajo Fin de Grado}}
%	\fancyhead[RO]{\text{Análisis de Series Temporales}}
%	\fancyfoot{} % clear footnote
%	\fancyfoot[RE]{\text{U-tad 2023-2024}} 
%	\fancyfoot[RO]{\text{MAIS}}
%	\fancyfoot[C]{\thepage}
%	\fancyfoot[L]{\text{Javier Coque}}
%	\renewcommand{\footrulewidth}{0.4pt}% default is 0pt
%}

%--------------------- NOTA IMPORTANTE -------------------------%
%---------------------------------------------------------------%
%Me acabo de crear arriba un estilo que se llama plain;  le podría haber llamado bocata. Para aplicar dicho header, basta con poner \pagestyle{plain} al principio del documento para que se aplique.


%El comando \leftmark sirve para asignar como texto el título de la sección. El caso es que solo funciona para secciones numeradas. Para que también funcione en secciones no numeradas deberemos usar el siguiente comando: \markboth{nombre_seccio_no_numerada}{nombre_seccion_no_numerada}. El primer parámetro indica lo que poner en páginas pares y el segundo en impares, así que se puede modificar a gusto propio.

%En caso de querer usar un estilo propio en alguna de las secciones, se puede hacer con \thispagestyle{fancy} y luego aplicar las customizaciones que quieras como pueden ser \fancyfoot[L]{\text{nuevo_foot_text}}. Al acabar, vuelve a poner el comnado \pagestyle{plain} para volver al otro estilo.

%Otra cosa que se puede hacer es crear  varios estilos de fancyhdr. Al igual que me he creado el estilo plain, me puedo cerar otros. 

%Lo que acabo de explicar está en mi TFG de matemáticas hecho.
%--------------------------------------------------------------------------%
%--------------------------------------------------------------------------%





% -----------------------  FOR MORE SUBSECTIONS ---------------------------%
%--------------------------------------------------------------------------%
% Define a custom heading for the fourth level
%%\newcommand{\subsubsubsection}[1]{\paragraph{#1}\mbox{}\\} %now I do have subsubsubsubsection
%%\setcounter{secnumdepth}{4} % how many sectioning levels to assign numbers to
%%\setcounter{tocdepth}{4} % Enable numbering for up to the fourth level
%--------------------------------------------------------------------------%
%--------------------------------------------------------------------------%





% --------------------==------  FOR MATH STUFF ----------------------------%
%--------------------------------------------------------------------------%
%%\usepackage{amsmath,amsfonts,amsthm, amssymb} % Para poder usar \mathbb
	%%\renewcommand{\qed}{\hfill\blacksquare} %blacksquare for demonstrations
%%\numberwithin{equation}{subsection} %this numbers the equatino along with the subsections
%%\usepackage{cancel} %to cancel numbers in  equations
%%\usepackage{mathtools} %to annotate brackets in equations
%--------------------------------------------------------------------------%
%--------------------------------------------------------------------------%





% ----------------------  FIGURES , TABLES & TIKZ  ------------------------%
%--------------------------------------------------------------------------%
\usepackage{graphicx}
    \graphicspath{ {./imgs/} }
%%\counterwithin{figure}{section}%figure number along with the section
%%\usepackage{subfigure} % for subfigures
%%\usepackage{wrapfig} % to wrap a figure among text
%%\usepackage{caption} % caption management in figures and tables
	%%\captionsetup[figure]{name=Figura}%Changes the default name Figure to Figura in figures
\usepackage{float} % for H command in figures and tables
%%\usepackage{listings} % para poder hacer uso de "listings" propios (p.ej. códigos)
%%	\lstset{
%%		language=Python,
%%		basicstyle=\ttfamily\small,
%%		keywordstyle=\bfseries\color{blue},
%%		stringstyle=\color{orange!70!black},
%%		commentstyle=\color{green!70!black},
%%		showstringspaces=false,
%%		frame=single,
%%		breaklines=true,
%%		captionpos=b
%%	}
\usepackage{longtable} % Para tablas muy largas
\usepackage{array} % for table's setting such as spacing within cells and 
				   % the border width
\renewcommand{\tablename}{Tabla} % %Changes the default name Table to Tabla in tables
%\newcolumntype{s}{>{\columncolor{blue!15}} c} %command for a specific type of column.
\renewcommand{\arraystretch}{3} % top and bottom spacing inside table's cells				   	
\usepackage{multirow} % Para agrupar varias filas en las tablas


\usepackage{tikz}
    \usetikzlibrary{shapes,arrows, positioning}
    \usetikzlibrary{patterns}
\usepackage{pgfplots}
    \usepgfplotslibrary{fillbetween}% color tikz's images
    \pgfplotsset{compat=1.17} %this can be change to compat = newest
%--------------------------------------------------------------------------%
%--------------------------------------------------------------------------%





% ---------------  TCOLORBOXES, COLORS AND OTHER DECORATIVES  -------------%
%--------------------------------------------------------------------------%
\usepackage{xcolor}  % color package to create colors. Additional table features added to xcolor package
%ATTENTION:NO SPACING IN COLOR RGB VALUES
	% ------------ RGB style -----------%
	\definecolor{redos}{RGB}{255,46,46}
	\definecolor{redcl}{RGB}{255,200,200}
	% ------------ HTML CODE style -----------%
	\definecolor{azulos}{HTML}{4291FD}
	\definecolor{azulcl}{HTML}{E8F3FF}
        \definecolor{verdeos}{RGB}{76,174,80}
        \definecolor{verdecl}{RGB}{231,244,233}
        \definecolor{morados}{RGB}{115, 91, 251}
        \definecolor{moradcl}{RGB}{228, 223, 252}
        \definecolor{orange}{RGB}{255, 197,0}
	
\usepackage[most]{tcolorbox} % package to create color boxes  like the note box

\newtcolorbox{atencion}{
	enhanced,
	breakable,
	fonttitle=\sffamily\bfseries,
	colbacktitle=redcl,
	coltitle=redos,
	title= Atencion:,
	colbacktitle=redcl,
	title code={
		\path[draw=redos,solid,line width=0.75mm]
		([xshift=0mm]title.south west) -- ([xshift=0mm]title.south east);
	},
	boxrule=0pt,frame hidden,
	borderline west={3pt}{0pt}{redos},
	colback=redcl,
}

\newtcolorbox{demo}[1]{
	enhanced,
	breakable,
	fonttitle=\sffamily\bfseries,
	colbacktitle=azulos,
	coltitle=azulos,
	title= #1,
	colbacktitle=azulcl,
	boxrule=0pt,frame hidden,
	borderline west={3pt}{0pt}{azulos},
	colback=azulcl,
	top = 10pt, %body margin
	before title={\vspace*{5pt}}, %title margin
	bottom = 8pt
}

\newtcolorbox{teorema}[1]{
	enhanced,
	breakable,
	fonttitle=\sffamily\bfseries,
	colbacktitle=azulcl,
	coltitle=azulos,
	title= #1,
	colbacktitle=azulcl,
	title code={
		\path[draw=azulos,solid,line width=0.75mm]
		([xshift=5mm]title.south west) -- ([xshift=0mm]title.south east);
	},
	boxrule=0pt,frame hidden,
	borderline west={3pt}{0pt}{azulos},
	colback=azulcl,
}

\newtcolorbox{ejemplo}[1]{
	enhanced,
	breakable,
	fonttitle=\sffamily\bfseries,
	%colbacktitle=azulos,
	coltitle=azulos,
	title=#1,
	%colbacktitle=azulcl,
	boxrule=0pt,frame hidden,
	borderline west={3pt}{0pt}{azulos},
	colback=azulcl,
}
\newtcolorbox{ejemplo2}[1]{
	enhanced,
	breakable,
	fonttitle=\sffamily\bfseries,
	colbacktitle=azulcl,
	coltitle=azulos,
	title= #1,
	colbacktitle=azulcl,
	title code={
		\path[draw=azulos,solid,line width=0.75mm]
		([xshift=0mm]title.south west) -- ([xshift=0mm]title.south east);
	},
	boxrule=0pt,frame hidden,
	borderline west={3pt}{0pt}{azulos},
	colback=azulcl,
}


\newtcolorbox{nota}{
	enhanced,
	breakable,
	fonttitle=\sffamily\bfseries,
	%colbacktitle=azulos,
	coltitle=azulos,
	title=Nota,
	colbacktitle=azulcl,
	boxrule=0pt,frame hidden,
	%borderline west={3pt}{0pt}{azulos},
	colback=white,
}

%\newtcolorbox{mathbox}{
%	colback = white,
%	colframe = azulos,
%}
%
%
%\newenvironment{matheq}{%
%	\begin{center}%
%		\begin{mathbox}%
%		}{%
%		\end{mathbox}%
%	\end{center}%
%}
%
%\newenvironment{matheq2}{%
%	\begin{center}%
%		\tcbox[colback = azulcl, colframe = azulos]%
%	}{%
%	\end{center}%
%}
%--------------------------------------------------------------------------%
%--------------------------------------------------------------------------%




% --------------------------  MISCELLANEOUS  ------------------------------%
%--------------------------------------------------------------------------%
\usepackage{comment} %to comment big sections with \begin{comment}...\end{comment}
\usepackage{enumitem} %for lists
\usepackage{url} %to place url with te command \url{https://...}
\usepackage{environ}% to create new environments

%--------------------------------------------------------------------------%
%--------------------------------------------------------------------------%




% --------------------------  REFERENCES  --------------------------------%
%-------------------------------------------------------------------------%
\usepackage[backend = biber, style = numeric]{biblatex} %change styple to apa if necessary
%\DeclareLanguageMapping{spanish}{spanish-apa}
\addbibresource{biblio.bib} % Fichero donde se incluyen las referencias
%---------------------------------------------------------------------------------------%
%---------------------------------------------------------------------------------------%








%---------------------------------------------------------------------------------------%
%---------------------------------------------------------------------------------------%
% -------------------------  NEW COMMANDS & NEW ENVIRONMENTS ---------------------------%
%---------------------------------------------------------------------------------------%
%---------------------------------------------------------------------------------------%




% ---------------------  GENERIC DOCUMENT SETTINGS ------------------------%
%--------------------------------------------------------------------------%
\renewcommand{\baselinestretch}{1.4}% Espaciado entre líneas
%\usepackage{setspace} %espaciado entre lineas de manera no tan precisa 
	%\onehalfspacin
\newcommand{\B}[1]{\textbf{#1}}
\newcommand{\slas}{$\backslash$}
\newcommand{\ti}{\emph} %it stands for text italic. The command \it is already taken.
\newcommand{\ul}{\underline}
\newcommand{\fn}{\footnote}
\newcommand{\fnm}{\footnotemark}
\newcommand{\fnt}{\footnotetext}
    \setlength{\footnotesep}{\baselineskip}%increase spacing between footnotes
% CHANGE THE DEFAULT NAME OF TOC AND LIST OF FIGURES/TABLES AT THE BEGINNING.
% i.e. when using \tableofcontents, \listoffigures, \listoftables in \begin{document}
%\renewcommand{\contentsname}{Tabla de contenidos}%cambiar titulo de \listofcontents (contents por defecto)
%\renewcommand{\listfigurename}{Lista de Figuras}
%\renewcommand{\listtablename}{Lista de Tablas}
%--------------------------------------------------------------------------%
%--------------------------------------------------------------------------%




	
% -------------------------------  MATH  ----------------------------------%
%--------------------------------------------------------------------------%
%This is to display inline math symbols as if they were block symbols. Shorten commands intead of
%using displaystyle... .
\newcommand{\Int}[2]{\displaystyle \int_{#1}^{#2}}
\newcommand{\Sum}[2]{\displaystyle \sum_{#1}^{#2}}
\newcommand{\Lim}[1]{\displaystyle \lim_{#1}}
\newcommand{\Frac}[2]{\displaystyle \frac{#1}{#2}}

%Make shorter the command to write the N of Naturals, Z of Integer,  R of Reals, Q of rational & C of complex
\newcommand{\N}{\mathbb{N}}
\newcommand{\Z}{\mathbb{Z}}
\newcommand{\Q}{\mathbb{Q}}
\newcommand{\R}{\mathbb{R}}
\newcommand{\C}{\mathbb{C}}

%to be able to write the text Var, Cov and Corr in math equations
\newcommand{\Var}{\mathrm{Var}}
\newcommand{\Cov}{\mathrm{Cov}}
\newcommand{\Corr}{\mathrm{Corr}}
%--------------------------------------------------------------------------%
%--------------------------------------------------------------------------%




% --------------------------  MISCELLANEOUS  ------------------------------%
%--------------------------------------------------------------------------%
\newcommand{\Code}[1]{\fcolorbox{code}{code}{#1}} %to write words meaning a command or code
	\definecolor{code}{RGB}{228, 229, 231}
	
% Para que el índice no sea clickable
%\makeatletter
%\let\Hy@linktoc\Hy@linktoc@none
%\makeatother

% To avoid word splitting between lines. Example: (first line) misce- (seccond line) llaneous.
%\pretolerance=10000
%--------------------------------------------------------------------------%
%--------------------------------------------------------------------------%




\title{3 Auditoría\\
\large Equipo auditor: ERP BI\\
\large Equipo auditado: Big Eye BA\\
\large Asignatura: Proyectos IV}

\author{
  \small{Juan Carlos Ávila}\fn{juan.avila@live.u-tad.com}
  \and
  \small{Javier Coque}\fn{javier.coque@live.u-tad.com}
  \and
  \small{Alejandro Gallego}\fn{alejandro.gallego@live.u-tad.com}
  \and
  \small{Chantal López}\fn{chantal.lopez@live.u-tad.com}
}

\date{\today}

\begin{document}
\maketitle


\pagenumbering{arabic}
\pagestyle{plain} % setting our default header and footer that we set with
% ÍNDICE DE CONTENIDOS
\setcounter{page}{1}

\section*{Introducción}

El equipo Big Eye BA presentó sus avances en el proyecto en ese último Sprint antes de la entrega al cliente final en un par de semanas. Aunque tengan tiempo de poder añadir o modificar pequeñas cosas, se esperaba que se nos presentase un producto casi terminado ---y así fue.

\section*{Resumen}

El equipo auditado estuvo comentando que no todo había ido como esperaban y algunos errores ---junto con sus soluciones--- fueron comentadas:

\begin{itemize}
    \item En primer lugar, mencionar que, inesperadamente, lo que más tiempo les llevó fue el refinamiento de los modelos en general, pero, especialmente el de Catastro. Concretamente, fue el preproceso, con el objetivo de simplificar los datos obtenidos de Catastro, lo que más problema dio. Al parecer, las clasificación entre \ti{edificio} y \ti{no edificio} seguía dando problemas (se clasificaba como edificio zonas que no eran un edificio). Como medida respecto que, cabe mencionar que es muy buena, es tener en cuenta aquellos "edificios" con \B{al  menos} un planta. Aunque todavía los resultados no son los esperados, nos parece un buen punto de partida del que partir para seguir mejorando el modelo y obtener unos resultados algo más satisfactorios.
    \item Respecto al modelo predictivo, TCN, en la último auditoría se comentó que se había elegido un sitio particular de Almería en donde no llovía casi. Esto porque las imágenes para estudiar el NDVI\fn{Índice de vegetación de diferencia normalizada.} son satelitales y los sitios muy nublados ocasionaban mucho ruido. Pese a que con esto se consigue mejorar el problema, no se consigue paliarlo del todo. La solución que aportan es no tener en cuenta estos datos, tratarlos como \Code{NaN} y hacer una interpolación cúbica. Aquí, nuestro equipo sugiere, para futuras ocasiones, que algo común y aparentemente más fácil es utilizar métodos como \Code{ffill} ó \Code{bfill} junto con \Code{fillna} para solventar el mismo problema. De todas maneras, su solución también parece correcta.
    \item Ligado con este punto anterior, también se comentó las dificultades que surgieron a la hora de desplegar los modelos de Pytorch en MLflow. Como respuesta a esto, se han desplegado en local. Algo que no les ha dado tiempo es la investigación del despliegue con la otra librería de \ti{machine learning}: Tensorflow. El cliente Pablo les sugiere hacerlo, pues es más fácil este proceso con la librería recientemente mencionada que con Pytorch.
\end{itemize}


\section*{Recomendaciones}

Como recomendaciones, el equipo ERP BI sugiere al equipo auditado que se reúnan para establecer que tareas pendientes son las más importantes para finalizar. Por ejemplo, el refinamiento de modelos parece ser más importante que mejorar el diseño y UX de la interfaz gráfica. Una vez ordenas las tareas por prioridad, que asignen personas a cada una de ellas según el tiempo y esfuerzo que en un primer momento se espere de ellas. 

Esto con el fin de poder completar un MVP lo más completo posible para la presentación del cliente final.

\section*{Conclusiones}

Se ha notado que el equipo, especialmente en este último Sprint ha estado más unido o mejor comunicado que en otros Sprints. Por ejemplo, el miembro del equipo Álvaro, no tuvo problema en comentar la parte de Javier. Esto es algo que queremos remarcar como un punto muy \B{\textcolor{verdeos}{positivo}}.

Por último ---y más importante--- queremos \ul{felicitar} al equipo por su gran trabajo. La dificultad del proyecto es muy alta y aunque los resultados son mejorables, se nota el esfuerzo, dedicación y compromiso por parte de todo el equipo. Nuestro equipo no tiene ninguna duda de que conseguir entregar un MVP que logre satisfacer las expectativas del cliente final.
\end{document}

